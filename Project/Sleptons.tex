\documentclass[twocolumn,a4paper,10pt]{article}
\usepackage[print,sort]{standalone}
\usepackage[T1]{fontenc}
\usepackage[utf8]{inputenc}
\usepackage[english]{babel}
\usepackage{graphicx,float}
%\usepackage{amssymb}
\usepackage{amsmath,cancel}
%\usepackage{mathrsfs}
\usepackage{epstopdf}
%\usepackage{subcaption}
%\usepackage{slashed}
%\usepackage{hhline}
%\usepackage[margin=1.2in]{geometry}
\usepackage[hidelinks]{hyperref}
%\usepackage{wrapfig}


\hfuzz=5pt
%\usepackage[dvips]{graphicx}

% Variable enumeration
\usepackage{enumerate}

% Use all allowed space
\addtolength{\hoffset}{-0.5cm}
\addtolength{\textwidth}{1.0cm}
\addtolength{\voffset}{-1.5cm}
\addtolength{\textheight}{3cm}
\setlength{\columnsep}{0.5cm}

% Remove Abstract titile for abstract
%\renewcommand{\abstractname}{}

\author{Even S. Håland}

\title{How light can sleptons be?}

\begin{document}

% Abstract construction for using both columns in two-column format
\twocolumn[
\begin{@twocolumnfalse}
  \maketitle
  \begin{abstract}
Several searches for production of sleptons has been performed in collider experiments such as the currently running LHC, and its predecessor LEP. In this article we investigate the limits on the slepton mass(es) set by these experiments, with special emphasis on how light sleptons are allowed to be with the current limits. We also take a look at the models used in these searches, how these compare to the MSSM, and how the limits from these searches can be related to the sleptons in the MSSM.
 \end{abstract}
  \vspace{5mm}
\end{@twocolumnfalse}
]

\section{Introduction}

The Minimal Supersymmetric Standard Model (MSSM) is the smallest possible supersymmetric 
of the Standard Model (SM). Roughly explained the MSSM introduces one superpartner for each 
SM particle (with some complications in the Higgs sector), and the SM particles and their 
superpartners (often called sparticles) differ by $\frac{1}{2}$ in spin. 
The sparticles we will focus on here are the superpartners of the SM leptons, which are scalar (i.e. 
spin-$0$) particles called \textit{sleptons}, and we will focus mainly on the first two generations of 
charged sleptons, i.e. selectrons ($\tilde{e}$) and smuons ($\tilde{\mu}$).  

In particle collider experiments, such as the Large Electron-Positron Collider (LEP) and the currently 
running Large Hadron Collider (LHC), several searches for production of sleptons have been done, but 
so far without any sign of their existence. The consequence of such "negative" searches is usually 
that new limits are put on the masses of the relevant sparticles (or other parameters).              

In this article we will try to summarize the current status of the limits on the selectron and smuon 
masses. However, since the MSSM is a quite complicated model with $105$ free parameters, one has to make 
assumptions and simplifications when setting limits, and these are (of course!) somewhat different from 
analysis to analysis. We will therefore also take a look at which assumptions are made in the different  
analyses. But before all this, let us look at how sleptons are described in the MSSM.          

\section{Sleptons in the MSSM}

As mentioned in the introduction the sleptons are the scalar superpartners of the SM leptons. In the 
SM there is an important difference between left- and right-handed chiral states, in that the 
left-handed leptons are organized in weak isospin doublets, while the right-handed ones are singlets 
(i.e. they do not transform under $SU(2)_L$). This mean that, when constructing a supersymmetric theory, 
we need to introduce separate superpartners for the left- and right-handed leptons. For this reason we 
talk about left- and right-handed sleptons ($\tilde{\ell}_L$ and $\tilde{\ell}_R$) even though they are 
scalar particles. 

This has some consequences when we are breaking SUSY. As we know, SUSY must be a broken symmetry, 
since otherwise particles and sparticles would have equal masses, meaning that SUSY would have been 
discovered a long time ago. For the first two generations of (charged) sleptons (neglecting the 
Yukawa coupling) the mass is given as
\begin{align}
m_{\tilde{\ell}}^2 = m_{\ell}^2 + (T_3 - Q\sin^2\theta_W)\cos 2\beta m_Z^2, 
\end{align}  
where $m_{\ell}$ is the  mass of the corresponding SM lepton, $T_3$ is weak isospin, $Q$ is 
electric charge, $\theta_W$ is the Weinberg angle, $\beta$ is given by the ration between the 
vacuum expectation values of the two Higgs doublets of the MSSM, and $m_Z$ is the mass of the $Z$-boson.
Interesting to notice is that $m_{\tilde{\ell}}$ depends on weak isospin, which is different for 
$\tilde{\ell}_L$ ($T_3 = -1/2$) and $\tilde{\ell}_R$ ($T_3 = 0$), meaning that their masses are 
different. The mass difference is given by 
\begin{align}
m_{\tilde{\ell}_L}^2 - m_{\tilde{\ell}_R}^2 = -\frac{1}{2}\cos 2\beta m_Z^2.  
\end{align}  
By convention we have $0 < \beta < \frac{\pi}{2}$, and it is (apparently, check this!) in the MSSM a 
common assumption that $\tan\beta > 1$, meaning that $\cos 2\beta < 0$, so 
\begin{align*}
m_{\tilde{\ell}_L}^2 > m_{\tilde{\ell}_R}^2. 
\end{align*} 

\section{Slepton production in particle colliders}

The MSSM is usually defined as conserving R-parity, given as  
\begin{align*}
R = (-1)^{2s + 3B + L}, 
\end{align*}
where $s$ is spin, $B$ is baryon number and $L$ is lepton number. This has the 
interesting consequences that sparticles will always be produced in pairs in particle colliders, 
the lightest sparticle (LSP) will be stable, and all other sparticles will (possibly via multiple steps) 
decay to the LSP. Conservation of R-parity (and some other quantum numbers) means that slepton searches 
target production of $\tilde{\ell}^+\tilde{\ell}^-$, and it is a very common assumption that the LSP is 
the lightest neutralino, $\tilde{\chi}_1^0$, which is an excellent candidate particle for dark matter. 
Often sleptons are assumed to decay directly to $\tilde{\chi}_1^0$, plus the corresponding SM lepton. 
The mass difference, 
\begin{align}
\Delta m = m_{\tilde{\ell}} - m_{\tilde{\chi}^0_1}, 
\label{eq:delta m}
\end{align}  
is in this case quite important, since it basically determines the momentum of the lepton, which is 
what you observe in the detector. Scenarios with small $\Delta m$ is in some experiments hard to study. 

In hadron colliders, such as the LHC, the cross section for slepton production is expected to be 
quite small, since the production of coloured (s)particles should be dominant in such machines. 
However, if coloured sparticles are sufficiently heavy, production of sleptons (and other electroweak 
sparticles) could be the leading SUSY production channel. At leading order a pair of sleptons can be 
produced through $q\bar{q}$ annihilation to a virtual $Z/\gamma$, which splits into 
$\tilde{\ell}^+\tilde{\ell}^-$ ($s$-channel). In lepton colliders there is a similar $s$-channel, only 
with $e^+e^-$ annihilation, but in addition there is also a $t$-channel with neutralino exchange 
available at leading order.  

\section{Slepton mass limits}

Now that we have introduced some theory and phenomenology concerning sleptons it is time to move into 
the more experimental details. We will mainly focus on searches done at LEP and the LHC, as the best 
current limits stems from these experiments. We will take a look at what the actual limits are, and 
which assumptions that are made in the various searches.    

\subsection{LEP}

The Large Electron-Positron collider (LEP) was a $27$ km $e^+e^-$ collider at CERN running between 
$1989$ and $2000$, and is still the most powerful lepton collider ever built, with a peak energy of 
$209$ GeV. Although this is much less than the energy at which the LHC collides protons, and the 
total delivered luminosity is much smaller than in the LHC, it is interesting to notice that the 
LEP experiments still has the most general limits on the masses of both selectrons and smuons.  

An absolute lower limit on the selectron masses, $m_{\tilde{e}_L}$ and $m_{\tilde{e}_R}$,  within the 
MSSM is set by the ALEPH experiment in ref. \cite{ALEPH:2002} to be  
\begin{align*}
m_{\tilde{e}_R} & > 73 \:\: \text{GeV}, \\
m_{\tilde{e}_L} & > 107 \:\: \text{GeV},   
\end{align*}       
assuming R-parity conservation, and that $\tilde{\chi}_1^0$ is the LSP. It is also assumed that 
scalar masses and gaugino masses are unified to $m_0$ and $m_{1/2}$ respectively at GUT scale, and 
that $\tan\beta > 1$, as mentioned previously. Finally, mixing between $\tilde{e}_L$ and $\tilde{e}_R$ 
is neglected. It is however noteworthy that these limits are for \textit{any} $\Delta m$ (see eq. 
\ref{eq:delta m}). 



\subsection{LHC}

\section{Simplified models vs the MSSM}

\section{Conclusions}

\begin{thebibliography}{9}

% Please take references when possible from SPIRES
% http://inspirehep.net/

\bibitem{ALEPH:2002}
  A. Heister et al. (2002), Phys.Lett. B544, 73-88. 
  \href{https://inspirehep.net/record/591226}{https://inspirehep.net/record/591226}
 
\bibitem{DELPHI:2003} 
  J. Abdallah et al. (2003), Eur.Phys.J.C31, 421-479. 
  \href{http://inspirehep.net/record/632738}{http://inspirehep.net/record/632738}	

  
\end{thebibliography}

\end{document}
